\documentclass[11pt]{article}
\title{%
Group 9 Project Memo\\\large
\color{white} blank line\\
\color{black}
Tracking Interconnected Facebook Links\\
with Graph Database Neo4j}

\date{Aug 2017}
\author{Lindiwe, Clifford, Thomas}

\usepackage[margin=1.5in]{geometry}
\usepackage{hyperref}
\hypersetup{
    colorlinks=true,
    linkcolor=blue,
    filecolor=magenta,      
    urlcolor=cyan,
}

\begin{document}
\maketitle
\pagenumbering{gobble}
\newpage
\tableofcontents
\newpage
\pagenumbering{arabic}
\section{Introduction}
\subsection{GitHub}
For now I have created a repository \url{https://github.com/TJ721988/SEgroup9}, hope that's correct.

\subsection{Project outline}
Graphically represent a database, with which the user can interact: manipulate structures, shapes and colours. 
Nodes represent individual people or groups, while links symbolise the connection(friends, like a page, join a group...)
\subsubsection{Back-end}
The back-end is a Neo4j database, starting with 3 basic tables; People, Groups, Pages
Groups and Pages are just lists of various groups, perhaps some properties too.
People to be updated every time a new person joins the network, the table links to the Groups and Pages through foreign keys,

\subsubsection{Front-end}
The front-end interface displays the results of queries graphically. 
Not sure if the data just needs to be displayed or we need to identify and show patterns within the data. 

\subsection{Todo list}
\subsubsection{All}
\begin{itemize}
\item Learn LaTeX basics
\item Get an understanding of Neo4j and Neo4j Browser
\item Refresh JSON and XML
\end{itemize}

\subsubsection{Back-end}
Clifford to set up the initial database
\subsubsection{Front-end}
Lindiwe and Thomas to work on data retrieval, display and manipulation, 
\\
\\
\\
Work load may need to be distributed, once we know which part will be the most work.

\subsection{Design}
The input should allow data to be entered into the database, as well as to allow graphic manipulation\\
The output should be a graphic representation of the connections in the network

\subsection{Management}
Once everyone is set up as a contributor on GitHub, anyone should be able to make changes and upload progress at any time. 
\begin{itemize}
\item weekly updates on whatsapp group
\item biweekly meetings, maybe before/after SE labs, to have more detailed discussion
\end{itemize}

\end{document}